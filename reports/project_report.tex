\documentclass[]{article}
\usepackage{lmodern}
\usepackage{amssymb,amsmath}
\usepackage{ifxetex,ifluatex}
\usepackage{fixltx2e} % provides \textsubscript
\ifnum 0\ifxetex 1\fi\ifluatex 1\fi=0 % if pdftex
  \usepackage[T1]{fontenc}
  \usepackage[utf8]{inputenc}
\else % if luatex or xelatex
  \ifxetex
    \usepackage{mathspec}
  \else
    \usepackage{fontspec}
  \fi
  \defaultfontfeatures{Ligatures=TeX,Scale=MatchLowercase}
\fi
% use upquote if available, for straight quotes in verbatim environments
\IfFileExists{upquote.sty}{\usepackage{upquote}}{}
% use microtype if available
\IfFileExists{microtype.sty}{%
\usepackage{microtype}
\UseMicrotypeSet[protrusion]{basicmath} % disable protrusion for tt fonts
}{}
\usepackage[margin=1in]{geometry}
\usepackage{hyperref}
\hypersetup{unicode=true,
            pdftitle={Examining the Effects of},
            pdfauthor={Taeil Goh, Greg Tozzi \& Max Ziff},
            pdfborder={0 0 0},
            breaklinks=true}
\urlstyle{same}  % don't use monospace font for urls
\usepackage{graphicx}
% grffile has become a legacy package: https://ctan.org/pkg/grffile
\IfFileExists{grffile.sty}{%
\usepackage{grffile}
}{}
\makeatletter
\def\maxwidth{\ifdim\Gin@nat@width>\linewidth\linewidth\else\Gin@nat@width\fi}
\def\maxheight{\ifdim\Gin@nat@height>\textheight\textheight\else\Gin@nat@height\fi}
\makeatother
% Scale images if necessary, so that they will not overflow the page
% margins by default, and it is still possible to overwrite the defaults
% using explicit options in \includegraphics[width, height, ...]{}
\setkeys{Gin}{width=\maxwidth,height=\maxheight,keepaspectratio}
\IfFileExists{parskip.sty}{%
\usepackage{parskip}
}{% else
\setlength{\parindent}{0pt}
\setlength{\parskip}{6pt plus 2pt minus 1pt}
}
\setlength{\emergencystretch}{3em}  % prevent overfull lines
\providecommand{\tightlist}{%
  \setlength{\itemsep}{0pt}\setlength{\parskip}{0pt}}
\setcounter{secnumdepth}{0}
% Redefines (sub)paragraphs to behave more like sections
\ifx\paragraph\undefined\else
\let\oldparagraph\paragraph
\renewcommand{\paragraph}[1]{\oldparagraph{#1}\mbox{}}
\fi
\ifx\subparagraph\undefined\else
\let\oldsubparagraph\subparagraph
\renewcommand{\subparagraph}[1]{\oldsubparagraph{#1}\mbox{}}
\fi

%%% Use protect on footnotes to avoid problems with footnotes in titles
\let\rmarkdownfootnote\footnote%
\def\footnote{\protect\rmarkdownfootnote}

%%% Change title format to be more compact
\usepackage{titling}

% Create subtitle command for use in maketitle
\providecommand{\subtitle}[1]{
  \posttitle{
    \begin{center}\large#1\end{center}
    }
}

\setlength{\droptitle}{-2em}

  \title{Examining the Effects of}
    \pretitle{\vspace{\droptitle}\centering\huge}
  \posttitle{\par}
    \author{Taeil Goh, Greg Tozzi \& Max Ziff}
    \preauthor{\centering\large\emph}
  \postauthor{\par}
      \predate{\centering\large\emph}
  \postdate{\par}
    \date{11/17/2020}


\begin{document}
\maketitle

\subsection{Abstract}\label{abstract}

\subsection{Introduction and Context}\label{introduction-and-context}

\subsubsection{The Children's Science
Center}\label{the-childrens-science-center}

The Children's Science Center (``the Center'') is a non-profit
organization headquartered in Fairfax, Virginia. Founded in TODO, the
Center's principal objective is to build the Northern Virginia region's
first interactive science center at a greenfield location on donated
land in Reston, Virginia. The capital campaign to raise fund to build
this facility is the Center's primary line of effort. A recent
partnership with the Commonwealth of Virginia's

Recognizing the long-term nature of the capital campaign and the need to
build interest and advocacy, the Center launched a series of
intermediate efforts beginning in TODO. The Center's first outreach
effort was the Museum without Walls, a program that delivered
science-based programs to schools around the region. In TODO, the Center
launched the Lab, a test facility located in a shopping mall sited 12
miles outside of the Washington, DC Beltway. The Lab hosted
approximately TODO guests per year until it was temporarily shut down in
response to the COVID-19 pandemic.

The Center funds operating expenses with Lab admissions, grants, and
philanthropic giving. The Center's Development Department is responsible
for developing and managing the latter two revenue streams.

\subsubsection{The Fundraising Surge}\label{the-fundraising-surge}

The Center conducts a fundraising surge toward the end of the calendar
year, consistent with the charitable giving cycle in the United States
{[}TODO: add reference{]}.

The Center expected to engage the majority of its stakeholders by email.
The Center viewed email marketing as repeated iterations of a three step
process. The Development Director expected that several rounds of
engagement would be necessary to achieve a conversion in the form of a
charitable gift. Within each individual round, the path to conversion
proceeded from receipt to opening to click-through using an embedded
link or button.

The Center introduced two major changes to its process ahead of the 2020
fundraising surge. In previous year-end campaigns, the Center had used
Constant Contact to send branded emails to stakeholders. In the 2020
campaign, the Center chose to switch to Mail Chimp's free tier. The
Center also engaged a consulting firm that applies machine learning to
the nonprofit space to predict top prospects from among the Center's
contact list and to predict affinities to aspects of the Center's
mission. The consulting firm's model requires a physical address as an
entering argument, and it was, therefore, only applied to that subset of
the Center's contact list for which physical addresses were entered.
Based on the consulting firm's output, the Center chose to engage a
subset of their stakeholder list with physical mail and to wall this
cohort off from subsequent email engagement.

\subsection{Research Questions}\label{research-questions}

The Center's development staff was most strongly interested in
conducting experiments to maximize the opening rates of the Center's
fundraising emails. The Development Director and her staff were keenly
aware that email opening was the critical first step in achieving
conversion. The follow-on actions---click though and conversion---were
areas of secondary interest for this study. With this in mind, we sought
to answer two specific questions posed by the Center.

\begin{enumerate}
\def\labelenumi{\arabic{enumi}.}
\item
  Is there a difference in email opening or click-through caused by
  using the Executive Director's name and title in the from line of the
  solicitation email versus using the Board Chair's name and title?
\item
  Is there a difference in email opening or click-through behavior
  caused by the choice of using one of two potential subject lines? The
  two subject lines considered were, ``You can be a Catalyst for STEM
  Learning 💡,'' and , ``Invest in the Power of STEM Learning 💡.''
\end{enumerate}

\subsubsection{Hypotheses}\label{hypotheses}

We claim expertise neither in non-profit email fundraising nor in the
preferences of the Center's stakeholders. The Center's development team
believed that there would be a significant difference in the opening
rates associated with the two email from lines considered, though the
staff did not have a sense of which would result in significantly higher
opening rates. Underpinning the staff's expectation that we would find a
significant difference was their belief that the Center's stakeholders
had a meaningful sense of the Center's leadership and governance
structure and would react differently when presented with emails from
either the Executive Director or the Board Chair.

Both of the subject lines considered were written by the Center's
development staff. The staff believed that subject lines could result in
significantly different opening rates but did not have a going-in belief
of which subject line would cause a stronger response.

\subsection{Experimental Design}\label{experimental-design}

\subsubsection{Overview}\label{overview}

The experiment was structured as two concurrant tests conducted on a
single subject cohort. With the cohort randomly assigned to four groups,
and with the from and subject lines denoted \(F_a\), \(F_b\), \(S_a\),
and \(S_b\), we assigned treatment combinations across balanced groups:

\begin{itemize}
\tightlist
\item
  Group 1 - \(F_a\) and \(S_a\)
\item
  Group 2 - \(F_a\) and \(S_b\)
\item
  Group 3 - \(F_b\) and \(S_a\)
\item
  Group 4 - \(F_b\) and \(S_b\)
\end{itemize}

This design allows us to explore heterogenous treatment effects.

Each group would recevied a tailored email containing its assigned
treatments sent through Mail Chimp. We intended to track opening and
click through using Mail Chimp's out-of-the-box analytics.

\subsubsection{Interface with Mail
Chimp}\label{interface-with-mail-chimp}

\subsubsection{Enrollment Process}\label{enrollment-process}

The Center provided an anonymized list of TODO stakeholders extracted
from its contact database. The individuals represented the Center's
complete contact list less a cohort of approximately 2,000 individuals
identifed by the Center for a special traditional mail campaign and less
a relatively small number of individuals walled off from fundraising
activities due to age, relationship to board members, and other similar
factors. The Center provided the output of the consulting firm's model
for individuals with mailing address on file. This output consisted of
buckted predictions of affinity for individuals to certain causes
related to the Center's mission---educational causes, children's causes,
cultural causes, and scientific causes.

We agreed to target a random sample of 2,000 idividuals for this
experiment to remain within the bounds of the Mail Chimp free tier. The
Center delivered one of four

\subsubsection{Randomization}\label{randomization}

\subsubsection{Accounting for
Non-Compliance}\label{accounting-for-non-compliance}

\subsection{Results}\label{results}

\subsection{Conclusions}\label{conclusions}


\end{document}
